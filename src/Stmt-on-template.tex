\pdfsuppresswarningpagegroup=1

% This file was converted to LaTeX by Writer2LaTeX ver. 1.6.1
% see http://writer2latex.sourceforge.net for more info
\documentclass[letterpaper]{article}
\usepackage[latin1]{inputenc}
\usepackage{amsmath}
\usepackage{amssymb,amsfonts,textcomp}
\usepackage[T1]{fontenc}
\usepackage[english]{babel}
\usepackage{color}
\usepackage{array}
\usepackage{supertabular}
\usepackage{hhline}
\usepackage{hyperref}
\usepackage{caption}
\hypersetup{pdftex, colorlinks=true, linkcolor=blue, citecolor=blue, filecolor=blue, urlcolor=blue, pdftitle=, pdfauthor=Barreiro; Andrea, pdfsubject=, pdfkeywords=}
\usepackage[pdftex]{graphicx}

% Text styles
\newcommand\textstyleInternetlink[1]{\textcolor[rgb]{0.019607844,0.3882353,0.75686276}{#1}}
\newcommand\textstylenormaltextrun[1]{#1}
\newcommand\textstyleeop[1]{#1}
% Page layout (geometry)
\setlength\voffset{-1in}
\setlength\hoffset{-1in}
\setlength\topmargin{0.5in}
\setlength\oddsidemargin{1in}
\setlength\textheight{8.079599in}
\setlength\textwidth{6.5in}
\setlength\footskip{0.9602in}
\setlength\headheight{0.5in}
\setlength\headsep{0.4602in}
% Footnote rule
\setlength{\skip\footins}{0.0469in}
\renewcommand\footnoterule{\vspace*{-0.0071in}\setlength\leftskip{0pt}\setlength\rightskip{0pt plus 1fil}\noindent\textcolor{black}{\rule{0.25\columnwidth}{0.0071in}}\vspace*{0.0398in}}
% Pages styles
\makeatletter

\newcommand{\VAR}[1] {$#1$}

\newcommand\ps@Standard{
  \renewcommand\@oddhead{}
  \renewcommand\@evenhead{\@oddhead}
  \renewcommand\@oddfoot{}
  \renewcommand\@evenfoot{\@oddfoot}
  \renewcommand\thepage{\arabic{page}}
}
\makeatother
\pagestyle{Standard}
\setlength\tabcolsep{1mm}
\renewcommand\arraystretch{1.3}
% List styles
\newcommand\liststyleWWNumii{%
\renewcommand\labelitemi{[F0B7?]}
\renewcommand\labelitemii{o}
\renewcommand\labelitemiii{[F0A7?]}
\renewcommand\labelitemiv{[F0B7?]}
}
\newcommand\liststyleWWNumiii{%
\renewcommand\labelitemi{[F0B7?]}
\renewcommand\labelitemii{o}
\renewcommand\labelitemiii{[F0A7?]}
\renewcommand\labelitemiv{[F0B7?]}
}
\newcommand\liststyleWWNumiv{%
\renewcommand\labelitemi{[F0B7?]}
\renewcommand\labelitemii{o}
\renewcommand\labelitemiii{[F0A7?]}
\renewcommand\labelitemiv{[F0B7?]}
}
\newcommand\liststyleWWNumi{%
\renewcommand\labelitemi{{\textperiodcentered}}
\renewcommand\labelitemii{o}
\renewcommand\labelitemiii{[F0A7?]}
\renewcommand\labelitemiv{[F0B7?]}
}

% This is a template for generating automatic reports for MUMTX.
% << Say more stuff here >>
% Here are variables that should be defined (will be referenced as \VAR{variable_name}):
%   (** means we need changes)
%   plan_name:       chamber+number short-hand (i.e. S2168 for TX Senate plan with this code on TLV website, 
%                    or DCN#### for a Dallas City Council plan
%   number_plans:    number of plans included in ensemble analysis
%   
%   plots_directory: directory where plots are to be found
%   report_directory: directory where low-res plan maps are found
%   election:        [PRES/SEN] which election to use for first introduction of violin plots
%   chamber**:         [USCD/TXSN/TXHD/DCN]
%   pretty_chamber:  
%   original_plan_mask:  ???
%   original_plan:   comparison plan?? (i.e. "current plan")
%   population_groups**: List of tuples (population_group, pretty_population_group) [TPOP,VAP,CVAP]
%   chamber:     provide as tuple [chamber, pretty_chamber]
%   plan_source: where plans come from (Texas Legislative Council, etc..)



\title{}
\author{Barreiro, Andrea}
%\date{2021-10-13}
\begin{document}
\textbf{\textit{MUM\_TX Statement on \VAR{plan_name}}}

\bigskip

\textbf{\textit{Math For Unbiased Maps TX (MUM\_TX) }}is\textbf{\textit{ }}an interdisciplinary, nonpartisan coalition
of Texas mathematicians, political scientists and philosophers working to ensure a fair and transparent redistricting
process. Our research concerns the development and application of ensemble sampling techniques, and in particular their
application to the current TX redistricting cycle. In brief, we use \textit{Markov Chain Monte Carlo }techniques to
generate a large number of random, legally valid maps which can then be used as an unbiased baseline to understand what
a typical map should look like. Conversely, when a proposed map is an outlier from the ensemble, this may be an
indication of gerrymandering.

\bigskip

We have applied our methods to the maps for \VAR{pretty_chamber} that have been made available by the \VAR{plan_source}. 
We generated a table of two important statistics that are commonly used by political scientists to assess
partisan gerrymandering: the mean-median score and partisan bias score. \ You can find the table at our webpage:
\href{http://www.smu.edu/Dedman/Research/Institutes-and-Centers/DCII/Scholarship/Research-Cluster-on-Political-Decision-Making/TXGerryWatch}{\textstyleInternetlink{www.smu.edu/Dedman/Research/Institutes-and-Centers/DCII/Scholarship/Research-Cluster-on-Political-Decision-Making/TXGerryWatch}}.



\bigskip

We compared the proposed map to an \textit{ensemble} of \VAR{number_plans} randomly-drawn maps. \ \ \ In Figure~\ref{fig:vsv-violin-with-comparison}, districts are ordered by the number of votes a Democratic
candidate for \VAR{pretty_chamber} would have received in the 2020 election, had voters used ``straight ticket'' voting. \ On
average, maps within our ensemble (blue dots) exhibit smoothly increasing vote shares as one moves from
Republican-leaning to Democratic-leaning districts. \ This smooth increase is the hallmark of an unbiased map. 
%\ \ But in the proposed map (red dots), the increase is highly disjointed, a clear sign of gerrymandering.

% ROBQ: plan number should include a prefix i.e. "DCN119299" rather than "119299"
\begin{figure}
\hspace*{-.5cm}
\includegraphics[width=7in]{\VAR{plots_directory}seats-voteshares-ensemble-comps-\VAR{chamber}-\VAR{original_plan_mask}-\VAR{plan}-\VAR{election}20.pdf}
\caption{Vote share vectors for proposed plan \VAR{plan} compared with an unbiased ensemble of randomly-generated plans. For each plan, districts are ordered by the number of votes a Democratic candidate would have received in the 2020 election, had voters used ``straight ticket'' voting. Blue ``violins'' show the the distribution of vote share values in the ensemble; blue horizontal lines mark the 1\% and 99\% percentile of each ordered statistic. The corresponding values for the proposed plan (red) and a comparison plan (green) are shown.}
\label{fig:vsv-violin-with-comparison}
\end{figure}

% ROBQ: Not a request, just an explanation for something I did: you can use only one dimension in \includegraphics. 
% e.g. use width but not height, or height and not width. Image will be scaled appropriately.
% Given that: I changed first few figs below to be 6.5-7 in width, removing "height" specification. 
% Some graphics were oversized.
% I will check white-space in next version, before I modify the entire document.

\bigskip

% ROBQ: I notice that the "packing" rectangle picks up too many districts. Only 3,5, and 14 are flagged as packed,
% judging from the later 2D histogram pics where districts are colord red/orange/yellow/purple based.
%
\begin{figure}[h!]
{\centering \includegraphics[width=6.5in]{\VAR{plots_directory}seats-voteshares-ensemble-enacted-\VAR{chamber}-\VAR{plan}-\VAR{election}20.pdf} \par}
\caption{As in Fig. \ref{fig:vsv-violin-with-comparison}, but now identifying (if applicable) distinct signatures of gerrymandering. A district was identified as ``cracked" if its ordered vote share was below the global median and also below the 1\% percentile of its respective ensemble position (CHECK); ``packed" if its ordered vote share was above the global median and also above the 99\% percentile of its respective ensemble position (CHECK).}
\label{fig:vsv-violin-with-cracking-packing}
\end{figure}

In Figure~\ref{fig:vsv-violin-with-cracking-packing}, we show the same data, but now identify two signatures of gerrymandering: ``cracking'' (where opposition voters are diluted to create safe districts for the majority) and ``packing'' (where these voters are concentrated to dilute their overall voting power).

\bigskip

% We note several specific features of the proposed plan. First, Democratic voters are disproportionately removed from a
% swath of districts in between 10 and 15 (District numbers are along the x-axis) that would be competitive in an
% unbiased map (a process known as ``cracking''), and placed into uncompetitive districts such as 7, 32, 20, 29, 35, 18,
% and 33 (a process known as ``packing''). \ Second, the list of outcomes between Districts 2 and 3 (a total of 13
% districts) is very nearly flat, which is a hallmark of maps created with the assistance of computer algorithms designed
% to automate the gerrymandering process. \ Finally, the predicted vote share between Districts 15 and 7 changes abruptly
% by about \textit{20 points, }with only 2 districts in between (28 and 34)\textit{ }{}-{}- this represents a ``wall''
% designed to protect legislators from changing voter opinions over time (see actual district numbers on the previous
% figure).

We also compute two common numbers that political scientists use to ``score'' maps. \ The first such number is called
the ``mean-median'' score: the difference in statewide vote percentage between the Republicans and Democrats required for them 
to win the majority of the chamber. The second such score is called the ``partisan bias'' score: \ the difference in the number of seats each party wins if
each were to earn 50\% of the vote. A positive score in either favors Democrats.


\begin{figure}[h!]
{\centering  \includegraphics[width=6in]{\VAR{plots_directory}mean-median-partisan-bias-ensemble-enacted-\VAR{chamber}-\VAR{plan}-\VAR{election}20.pdf}  \\
 \includegraphics[width=5.8in]{\VAR{plots_directory}partisan-metrics-2D-\VAR{chamber}-\VAR{plan}-\VAR{election}20.pdf} \par}
\caption{(Top) The mean-median (left) and partisan bias (right) scores for the ensemble, illustrated as a histogram. The mean-median score is normalized so that it is equal to the difference in vote share that is required for each party to win a majority of seats in the chamber. The partisan bias score is normalized so that it is equal to the difference in seats that each party wins if it wins 50\% of the vote. The corresponding values for the proposed plan (purple) and a comparison plan (green) are also shown. (Bottom) The same data, displayed as a 2D histogram (EXPLAIN SMOOTHING HERE). The proposed plan is shown in red.}
\label{fig:mm-and-pb-1d-and-2d-histograms}
\end{figure}


\bigskip


Of course, no plan is going to be perfectly aligned with the ensemble, so just how gerrymandered is this plan? \ A
little? \ A lot? \ An extreme amount? \ This question can be answered using statistics, by comparing each score above
to the \textit{distribution }of those scores within the \VAR{number_plans} - map ensemble. \ This is done in Figure~\ref{fig:mm-and-pb-1d-and-2d-histograms}.  The second row shows the joint distribution of mean-median and partisan bias. 

\begin{figure}
{\centering \includegraphics[width=6.5in,height=3.5in]{\VAR{plots_directory}violin-plot-\VAR{chamber}-\VAR{original_plan_mask}-\VAR{plan}-BH-VAP.pdf} \par}
\caption{Minority population percentage vectors for the proposed plan \VAR{plan} compared with an unbiased ensemble of randomly-generated plans. For each plan, districts are ordered by the percentage of the voting age population that is Black or Hispanic. Green ``violins" show the the distribution of percentages in the ensemble; blue horizontal lines mark the 1\% and 99\% percentile of each ordered statistic. The corresponding values for the proposed plan (red) and a comparison plan (green) are shown.}
\label{fig:bhvap-violins-with-comparison}
\end{figure}

\bigskip

In Figure~\ref{fig:bhvap-violins-with-comparison} we present another ``violin'' plot, but with districts sorted according to the
the fraction of the voting age population that is Black + Hispanic. 

\bigskip

We next ask the question ``How many Districts have an BHVAP over 50\%?'' (or 60\%, or 70\%, etc..). The
BHVAP is the combined Black and Hispanic voting age population. These histograms show the values for the ensemble, and
the value for the Proposed map is shown in red. In Figure~\ref{fig:num-dists-bhvap-perc} we show the number of districts that have BHVAP above 50\% and 70\%; that is the number of districts that are majority-minority, vs. the number of Districts that are
\textbf{\textit{overwhelmingly}} majority-minority. 

\begin{figure}
\begin{flushleft}
\tablefirsthead{}
\tablehead{}
\tabletail{}
\tablelasttail{}
\begin{supertabular}{|l|l|}
\hline
\includegraphics[width=3.25in]{\VAR{plots_directory}hist-small-\VAR{chamber}-\VAR{original_plan}-\VAR{plan}-BH-VAP-0.5.pdf}  &
\includegraphics[width=3.25in]{\VAR{plots_directory}hist-small-\VAR{chamber}-\VAR{original_plan}-\VAR{plan}-BH-VAP-0.7.pdf} \\
\hline
\end{supertabular}
\end{flushleft}
\caption{Number of districts that have a percentage of minority voting age population (specifically, Black and Hispanic) exceeding 50\% (left) and 70\% (right). An excessive number of heavily minority districts may be a signature of vote dilution, if it reduces the overall voting strength of the minority group. }
\label{fig:num-dists-bhvap-perc}
\end{figure}
\maxdeadcycles=1000
\extrafloats{100}

%Here is what we observe:

%\liststyleWWNumi
%\begin{itemize}
%\item The number of majority-minority districts in the Proposed plan (\VAR{plan_name}) is much lower than the typical %value in the
%ensemble (14 vs. 16-17).
%\item The number of overwhelmingly majority-minority districts is higher than the typical value in the ensemble (10 vs.
%6).
%\item This suggest that minority voters have been packed into a small number of districts, at the cost of reducing the
%total number of districts in which they may be able to elect representatives of their choice.
%\end{itemize}

%In summary, \VAR{plan_name} fails Texas voters by gerrymandering along both partisan and racial dimensions. First, this map
%artificially reduces the competitiveness of a large number of districts: \textit{at most} 3 out of 38 districts might
%charitably be viewed as competitive in a general election. The map also inflates the advantage to the Republican Party,
%in comparison to a typical unbiased map; an unbiased map would be closely balanced between the parties (and far more
%reflective of the views of actual Texas voters, who voted 54\% to 46\% in the 2020 election). Second, this map packs
%Black and Hispanic voters into a relatively small number of overwhelmingly minority districts, at the cost of reducing
%the total number of districts which effectively perform for minorities. We urge legislators to go back to the drawing
%board and return with a map that is fair to Texas voters.


\clearpage

\bigskip

\section*{Appendix: All Plots}

\begin{figure}
{\centering \includegraphics[width=6.5in]{\VAR{report_directory}report_\VAR{plan_name}_diff_map.png} \par}
\caption{\href{\VAR{url_root}report_\VAR{plan_name}_diff_map.pdf}{Click here} to download a higher resolution version of this image.}
\label{fig:plan_map}
\end{figure}

\begin{figure}
{\centering \includegraphics[width=6.5in]{\VAR{plots_directory}seats-voteshares-ensemble-comps-\VAR{chamber}-\VAR{original_plan_mask}-\VAR{plan}-PRES20.pdf} \par}
{\centering \includegraphics[width=6.5in{\VAR{plots_directory}seats-voteshares-ensemble-comps-\VAR{chamber}-\VAR{original_plan_mask}-\VAR{plan}-SEN20.pdf} \par}
\caption{Vote-share vectors for all districts, ordered by increasing Democratic vote share, for the 2020 Presidential (top) and US Senate (bottom) elections. Along with the ensemble (blue violins), vote-share vectors for the proposed (red) and current (green) plans are shown.}
\label{fig:voteshares-ensem-comps}
\end{figure}

\begin{figure}
{\centering \includegraphics[width=6.5in,height=3.5in]{\VAR{plots_directory}seats-voteshares-ensemble-enacted-\VAR{chamber}-\VAR{plan}-PRES20.pdf} \par}
\caption{Vote-share vectors for all districts, ordered by increasing Democratic vote share, for the 2020 Presidential election. Along with the ensemble (blue violins), the vote-share vector for the proposed (red) plan is shown. }
\label{fig:seats-voteshares-ensemble-enacted-PRES20}
\end{figure}

\begin{figure}
{\centering \includegraphics[width=6in]{\VAR{plots_directory}mean-median-partisan-bias-ensemble-enacted-\VAR{chamber}-\VAR{plan}-PRES20.pdf} \par}
\caption{The mean-median (left) and partisan bias (right) scores for the ensemble, illustrated as a histogram, using vote shares from the 2020 Presidential election. The mean-median score is normalized so that it is equal to the difference in vote share that is required for each party to win a majority of seats in the chamber. The partisan bias score is normalized so that it is equal to the difference in seats that each party wins if it wins 50\% of the vote. The corresponding values for the proposed plan (purple) and the ensemble median (green) are also shown. }
\label{fig:mean-median-partisan-bias-ensemble-enacted-PRES20}
\end{figure}

\begin{figure}
{\centering \includegraphics[width=6in]{\VAR{plots_directory}partisan-metrics-2D-\VAR{chamber}-\VAR{plan}-PRES20.pdf} \par}
\caption{The mean-median and partisan bias scores for the ensemble, displayed as a 2D histogram, using vote shares from the 2020 Presidential election. The proposed plan is shown in red.}
\label{fig:partisan-metrics-2D-PRES20}
\end{figure}

\begin{figure}
{\centering \includegraphics[width=6.5in,height=3.5in]{\VAR{plots_directory}seats-voteshares-ensemble-enacted-\VAR{chamber}-\VAR{plan}-SEN20.pdf} \par}
\caption{Vote-share vectors for all districts, ordered by increasing Democratic vote share, for the 2020 US Senate election. Along with the ensemble (blue violins), vote-share vectors for the proposed (red) plan is shown.}
\label{fig:seats-voteshares-ensemble-enacted-SEN20}
\end{figure}

\begin{figure}
{\centering \includegraphics[width=6in]{\VAR{plots_directory}mean-median-partisan-bias-ensemble-enacted-\VAR{chamber}-\VAR{plan}-SEN20.pdf} \par}
\caption{The mean-median (left) and partisan bias (right) scores for the ensemble, illustrated as a histogram, using vote shares from the 2020 US Senate election. The mean-median score is normalized so that it is equal to the difference in vote share that is required for each party to win a majority of seats in the chamber. The partisan bias score is normalized so that it is equal to the difference in seats that each party wins if it wins 50\% of the vote. The corresponding values for the proposed plan (purple) and a comparison plan (green) are also shown. }
\label{fig:mean-median-partisan-bias-ensemble-enacted-SEN20}
\end{figure}

\begin{figure}
{\centering \includegraphics[width=6in]{\VAR{plots_directory}partisan-metrics-2D-\VAR{chamber}-\VAR{plan}-SEN20.pdf} \par}
\caption{The mean-median and partisan bias scores for the ensemble, displayed as a 2D histogram, using vote shares from the 2020 US Senate election. The proposed plan is shown in red.}
\label{fig:partisan-metrics-2D-SEN20}
\end{figure}

\BLOCK{ for (population_group,pretty_population_group) in population_groups }

\begin{figure}
{\centering \includegraphics[width=6.5in,height=3.5in]{\VAR{plots_directory}violin-plot-\VAR{chamber}-\VAR{original_plan_mask}-\VAR{plan}-B-\VAR{population_group}.pdf} \par}
\caption{Black population percentage vectors for the proposed plan \VAR{plan} compared with an unbiased ensemble of randomly-generated plans. For each plan, districts are ordered by the percentage of the \VAR{pretty_population_group} that is Black. Orange ``violins" show the the distribution of percentages in the ensemble; blue horizontal lines mark the 1\% and 99\% percentile of each ordered statistic. The corresponding values for the proposed plan (red) and a comparison plan (green) are shown.}
\label{fig:violin-plot-B-\VAR{population_group}}
\end{figure}

\begin{figure}
{\centering \includegraphics[width=6.5in,height=3.5in]{\VAR{plots_directory}violin-plot-\VAR{chamber}-\VAR{original_plan_mask}-\VAR{plan}-H-\VAR{population_group}.pdf} \par}
\caption{Hispanic population percentage vectors for the proposed plan \VAR{plan} compared with an unbiased ensemble of randomly-generated plans. For each plan, districts are ordered by the percentage of the \VAR{pretty_population_group} that is Hispanic. Violet ``violins" show the the distribution of percentages in the ensemble; blue horizontal lines mark the 1\% and 99\% percentile of each ordered statistic. The corresponding values for the proposed plan (red) and a comparison plan (green) are shown.}
\label{fig:violin-plot-H-\VAR{population_group}}
\end{figure}

\begin{figure}
{\centering \includegraphics[width=6.5in,height=3.5in]{\VAR{plots_directory}violin-plot-\VAR{chamber}-\VAR{original_plan_mask}-\VAR{plan}-BH-\VAR{population_group}.pdf} \par}
\caption{Minority population percentage vectors for the proposed plan \VAR{plan} compared with an unbiased ensemble of randomly-generated plans. For each plan, districts are ordered by the percentage of the \VAR{pretty_population_group} that is Black or Hispanic. Green ``violins" show the the distribution of percentages in the ensemble; blue horizontal lines mark the 1\% and 99\% percentile of each ordered statistic. The corresponding values for the proposed plan (red) and a comparison plan (green) are shown.}
\label{fig:violin-plot-BH-\VAR{population_group}}
\end{figure}

\begin{figure}
{\centering \includegraphics[width=6.5in,height=3.5in]{\VAR{plots_directory}violin-plot-\VAR{chamber}-\VAR{original_plan_mask}-\VAR{plan}-NW-\VAR{population_group}.pdf} \par}
\caption{Non-white population percentage vectors for the proposed plan \VAR{plan} compared with an unbiased ensemble of randomly-generated plans. For each plan, districts are ordered by the percentage of the \VAR{pretty_population_group} that is non-white. Orange ``violins" show the the distribution of percentages in the ensemble; blue horizontal lines mark the 1\% and 99\% percentile of each ordered statistic. The corresponding values for the proposed plan (red) and a comparison plan (green) are shown.}
\label{fig:violin-plot-NW-\VAR{population_group}}
\end{figure}
\BLOCK{ endfor }

% ROBQ: I modified dimensions in the next block
\BLOCK{ for (population_group,pretty_population_group) in population_groups }
\begin{figure}[h]
\includegraphics[width=3.1in]{\VAR{plots_directory}hist-small-\VAR{chamber}-\VAR{original_plan}-\VAR{plan}-B-\VAR{population_group}-0.3.pdf}
\includegraphics[width=3.1in]{\VAR{plots_directory}hist-small-\VAR{chamber}-\VAR{original_plan}-\VAR{plan}-B-\VAR{population_group}-0.4.pdf}\\
\includegraphics[width=3.1in]{\VAR{plots_directory}hist-small-\VAR{chamber}-\VAR{original_plan}-\VAR{plan}-B-\VAR{population_group}-0.5.pdf}
\caption{Number of districts that have a percentage of the Black \VAR{pretty_population_group} exceeding 30\% (top left), 40\% (top right), and 50\% (bottom). The histogram shows values for the ensemble; the corresponding values for the proposed plan (red) and a comparison plan (green) are also shown.}
\label{fig:hist-small-B-\VAR{population_group}}
\end{figure}

\begin{figure}[h]
\includegraphics[width=3.1in]{\VAR{plots_directory}hist-small-\VAR{chamber}-\VAR{original_plan}-\VAR{plan}-H-\VAR{population_group}-0.5.pdf}
\includegraphics[width=3.1in]{\VAR{plots_directory}hist-small-\VAR{chamber}-\VAR{original_plan}-\VAR{plan}-H-\VAR{population_group}-0.55.pdf}\\
\includegraphics[width=3.1in]{\VAR{plots_directory}hist-small-\VAR{chamber}-\VAR{original_plan}-\VAR{plan}-H-\VAR{population_group}-0.6.pdf}
\includegraphics[width=3.1in]{\VAR{plots_directory}hist-small-\VAR{chamber}-\VAR{original_plan}-\VAR{plan}-H-\VAR{population_group}-0.65.pdf}
\caption{Number of districts that have a percentage of the Hispanic \VAR{pretty_population_group} exceeding 50\% (top left), 55\% (top right), 60\% (bottom left), and 65\% (bottom right). The histogram shows values for the ensemble; the corresponding values for the proposed plan (red) and a comparison plan (green) are also shown.}
\label{fig:hist-small-H-\VAR{population_group}}
\end{figure}

\begin{figure}[h]
\includegraphics[width=3.1in]{\VAR{plots_directory}hist-small-\VAR{chamber}-\VAR{original_plan}-\VAR{plan}-BH-\VAR{population_group}-0.5.pdf}
\includegraphics[width=3.1in]{\VAR{plots_directory}hist-small-\VAR{chamber}-\VAR{original_plan}-\VAR{plan}-BH-\VAR{population_group}-0.6.pdf}\\
\includegraphics[width=3.1in]{\VAR{plots_directory}hist-small-\VAR{chamber}-\VAR{original_plan}-\VAR{plan}-BH-\VAR{population_group}-0.7.pdf}
\caption{Number of districts that have a percentage of the Black or Hispanic \VAR{pretty_population_group} exceeding 50\% (top left), 60\% (top right), and 70\% (bottom). The histogram shows values for the ensemble; the corresponding values for the proposed plan (red) and a comparison plan (green) are also shown.}
\label{fig:hist-small-BH-\VAR{population_group}}
\end{figure}

\begin{figure}[h]
\includegraphics[width=3.1in]{\VAR{plots_directory}hist-small-\VAR{chamber}-\VAR{original_plan}-\VAR{plan}-NW-\VAR{population_group}-0.5.pdf}
\includegraphics[width=3.1in]{\VAR{plots_directory}hist-small-\VAR{chamber}-\VAR{original_plan}-\VAR{plan}-NW-\VAR{population_group}-0.6.pdf}\\
\includegraphics[width=3.1in]{\VAR{plots_directory}hist-small-\VAR{chamber}-\VAR{original_plan}-\VAR{plan}-NW-\VAR{population_group}-0.7.pdf}
\caption{Number of districts that have a percentage of the non-white \VAR{pretty_population_group} exceeding 50\% (top left), 60\% (top right), and 70\% (bottom). The histogram shows values for the ensemble; the corresponding values for the proposed plan (red) and a comparison plan (green) are also shown.}
\label{fig:hist-small-NW-\VAR{population_group}}
\end{figure}
\BLOCK{ endfor }

% VRA Metrics from MGGG
\BLOCK{ if have_vra }
\section*{VRA}

\BLOCK{ for model in ['equal', 'state'] }
\BLOCK{ for racial_group in ['BA', 'B', 'HA', 'H', 'BHO', 'BH'] }
\begin{figure}[h]
\includegraphics[width=6.5in,height=3.5in]{\VAR{plots_directory}violin-plot-\VAR{chamber}-\VAR{original_plan_mask}-\VAR{plan}-\VAR{racial_group}-CVAP-mggg_eff_\VAR{model}.pdf}
\caption{Come back to this later: need a report with these pics}
\label{fig:}
\end{figure}

\BLOCK{ endfor }
\BLOCK{ endfor }

\BLOCK{ for model in ['equal', 'state'] }
\BLOCK{ for racial_group in ['BA', 'B', 'HA', 'H', 'BHO', 'BH'] }
\begin{figure}[h]
\includegraphics[width=3.111in,height=2.0555in]{\VAR{plots_directory}hist-small-\VAR{chamber}-\VAR{original_plan}-\VAR{plan}-\VAR{racial_group}-CVAP-mggg_eff_\VAR{model}-0.3.pdf}
\includegraphics[width=3.111in,height=2.0555in]{\VAR{plots_directory}hist-small-\VAR{chamber}-\VAR{original_plan}-\VAR{plan}-\VAR{racial_group}-CVAP-mggg_eff_\VAR{model}-0.4.pdf}
\caption{}
\label{fig:}
\end{figure}

\begin{figure}[h]
\includegraphics[width=3.111in,height=2.0555in]{\VAR{plots_directory}hist-small-\VAR{chamber}-\VAR{original_plan}-\VAR{plan}-\VAR{racial_group}-CVAP-mggg_eff_\VAR{model}-0.5.pdf}
\includegraphics[width=3.111in,height=2.0555in]{\VAR{plots_directory}hist-small-\VAR{chamber}-\VAR{original_plan}-\VAR{plan}-\VAR{racial_group}-CVAP-mggg_eff_\VAR{model}-0.6.pdf}
\caption{}
\label{fig:}
\end{figure}

\begin{figure}[h]
\includegraphics[width=3.111in,height=2.0555in]{\VAR{plots_directory}hist-small-\VAR{chamber}-\VAR{original_plan}-\VAR{plan}-\VAR{racial_group}-CVAP-mggg_eff_\VAR{model}-0.7.pdf}
\caption{}
\label{fig:}
\end{figure}
\BLOCK{ endfor }
\BLOCK{ endfor }
\BLOCK{ endif }  % IF using VRA metrics from MGGG

\BLOCK{ for (population_group,pretty_population_group) in population_groups }
\begin{figure}[h]{\centering \includegraphics[width=7.3307in,height=3.9472in]{\VAR{plots_directory}racial-political-deviations-with-ensemble-\VAR{chamber}-\VAR{plan}-PRES20-B-\VAR{population_group}.pdf} \par}
\caption{Correlation between Democratic vote share and Black \VAR{pretty_population_group}, illustrated as deviations from the jurisdiction-wide means. For example: suppose the jurisdiction-wide (i.e. statewide or citywide) Democratic vote share is 60\%, and the jurisdiction-wide Black share of \VAR{pretty_population_group} is 35\%. Then a district with 52\% Democratic vote share and 36\% Black \VAR{pretty_population_group} will have Democratic Voter Deviation 0.52-0.6 = -0.08 and Minority Population Deviation of 0.36-0.35 = +0.01. Democratic vote shares are from the 2020 Presidential election. Values for districts in the ensemble are shown as a 2D histogram; values for the proposed plan are superimposed and are colored by whether they show signatures of partisan gerrymandering: cracked (yellow), packed (orange), stuffed (purple), or unclassified (red). }
\label{fig:rpd-with-ens-PRES20-B-\VAR{population_group}}
\end{figure}

\begin{figure}[h]{\centering \includegraphics[width=7.3307in,height=3.9472in]{\VAR{plots_directory}racial-political-deviations-with-ensemble-\VAR{chamber}-\VAR{plan}-PRES20-H-\VAR{population_group}.pdf} \par}
\caption{Correlation between Democratic vote share and Hispanic \VAR{pretty_population_group}, illustrated as deviations from the jurisdiction-wide means
(see Fig. \ref{fig:rpd-with-ens-PRES20-B-\VAR{population_group}} for an explanation of how this is defined). Democratic vote shares are from the 2020 Presidential election.  Values for districts in the ensemble are shown as a 2D histogram; values for the proposed plan are superimposed and are colored by whether they show signatures of partisan gerrymandering: cracked (yellow), packed (orange), stuffed (purple), or unclassified (red). }
\label{fig:rpd-with-ens-PRES20-H-\VAR{population_group}}
\end{figure}

\begin{figure}[h]{\centering \includegraphics[width=7.3307in,height=3.9472in]{\VAR{plots_directory}racial-political-deviations-with-ensemble-\VAR{chamber}-\VAR{plan}-PRES20-BH-\VAR{population_group}.pdf} \par}
\caption{Correlation between Democratic vote share and Black and Hispanic \VAR{pretty_population_group}, illustrated as deviations from the jurisdiction-wide means
(see Fig. \ref{fig:rpd-with-ens-PRES20-B-\VAR{population_group}} for an explanation of how this is defined). Democratic vote shares are from the 2020 Presidential election.  Values for districts in the ensemble are shown as a 2D histogram; values for the proposed plan are superimposed and are colored by whether they show signatures of partisan gerrymandering: cracked (yellow), packed (orange), stuffed (purple), or unclassified (red). }
\label{fig:rpd-with-ens-PRES20-BH-\VAR{population_group}}
\end{figure}

\begin{figure}[h]{\centering \includegraphics[width=7.3307in,height=3.9472in]{\VAR{plots_directory}racial-political-deviations-with-ensemble-\VAR{chamber}-\VAR{plan}-PRES20-NW-\VAR{population_group}.pdf} \par}
\caption{Correlation between Democratic vote share and non-white \VAR{pretty_population_group}, illustrated as deviations from the jurisdiction-wide means
(see Fig. \ref{fig:rpd-with-ens-PRES20-B-\VAR{population_group}} for an explanation of how this is defined). Democratic vote shares are from the 2020 Presidential election.  Values for districts in the ensemble are shown as a 2D histogram; values for the proposed plan are superimposed and are colored by whether they show signatures of partisan gerrymandering: cracked (yellow), packed (orange), stuffed (purple), or unclassified (red). }
\label{fig:rpd-with-ens-PRES20-NW-\VAR{population_group}}
\end{figure}
\BLOCK{ endfor }

\BLOCK{ for (population_group,pretty_population_group) in population_groups }
\begin{figure}[h]{\centering \includegraphics[width=7.3307in,height=3.9472in]{\VAR{plots_directory}racial-political-deviations-with-ensemble-\VAR{chamber}-\VAR{plan}-SEN20-B-\VAR{population_group}.pdf} \par}
\caption{Correlation between Democratic vote share and Black \VAR{pretty_population_group}, illustrated as deviations from the jurisdiction-wide means. For example: suppose the jurisdiction-wide (i.e. statewide or citywide) Democratic vote share is 60\%, and the jurisdiction-wide Black share of \VAR{pretty_population_group} is 35\%. Then a district with 52\% Democratic vote share and 36\% Black \VAR{pretty_population_group} will have Democratic Voter Deviation 0.52-0.6 = -0.08 and Minority Population Deviation of 0.36-0.35 = +0.01. Democratic vote shares are from the 2020 US Senate election. Values for districts in the ensemble are shown as a 2D histogram; values for the proposed plan are superimposed and are colored by whether they show signatures of partisan gerrymandering: cracked (yellow), packed (orange), stuffed (purple), or unclassified (red). }
\label{fig:rpd-with-ens-SEN20-B-\VAR{population_group}}
\end{figure}

\begin{figure}[h]{\centering \includegraphics[width=7.3307in,height=3.9472in]{\VAR{plots_directory}racial-political-deviations-with-ensemble-\VAR{chamber}-\VAR{plan}-SEN20-H-\VAR{population_group}.pdf} \par}
\caption{Correlation between Democratic vote share and Hispanic \VAR{pretty_population_group}, illustrated as deviations from the jurisdiction-wide means
(see Fig. \ref{fig:rpd-with-ens-SEN20-B-\VAR{population_group}} for an explanation of how this is defined). Democratic vote shares are from the 2020 US Senate election.  Values for districts in the ensemble are shown as a 2D histogram; values for the proposed plan are superimposed and are colored by whether they show signatures of partisan gerrymandering: cracked (yellow), packed (orange), stuffed (purple), or unclassified (red).}
\label{fig:rpd-with-ens-SEN20-H-\VAR{population_group}}
\end{figure}

\begin{figure}[h]{\centering \includegraphics[width=7.3307in,height=3.9472in]{\VAR{plots_directory}racial-political-deviations-with-ensemble-\VAR{chamber}-\VAR{plan}-SEN20-BH-\VAR{population_group}.pdf} \par}
\caption{Correlation between Democratic vote share and the Black and Hispanic \VAR{pretty_population_group}, illustrated as deviations from the jurisdiction-wide means
(see Fig. \ref{fig:rpd-with-ens-SEN20-B-\VAR{population_group}} for an explanation of how this is defined). Democratic vote shares are from the 2020 US Senate election.  Values for districts in the ensemble are shown as a 2D histogram; values for the proposed plan are superimposed and are colored by whether they show signatures of partisan gerrymandering: cracked (yellow), packed (orange), stuffed (purple), or unclassified (red).}
\label{fig:rpd-with-ens-SEN20-BH-\VAR{population_group}}
\end{figure}

\begin{figure}[h]{\centering \includegraphics[width=7.3307in,height=3.9472in]{\VAR{plots_directory}racial-political-deviations-with-ensemble-\VAR{chamber}-\VAR{plan}-SEN20-NW-\VAR{population_group}.pdf} \par}
\caption{Correlation between Democratic vote share and non-white \VAR{pretty_population_group}, illustrated as deviations from the jurisdiction-wide means
(see Fig. \ref{fig:rpd-with-ens-SEN20-B-\VAR{population_group}} for an explanation of how this is defined). Democratic vote shares are from the 2020 US Senate election.  Values for districts in the ensemble are shown as a 2D histogram; values for the proposed plan are superimposed and are colored by whether they show signatures of partisan gerrymandering: cracked (yellow), packed (orange), stuffed (purple), or unclassified (red).}
\label{fig:rpd-with-ens-SEN20-NW-\VAR{population_group}}
\end{figure}
\BLOCK{ endfor }

\end{document}
