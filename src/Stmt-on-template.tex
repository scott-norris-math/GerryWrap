\pdfsuppresswarningpagegroup=1

% This file was converted to LaTeX by Writer2LaTeX ver. 1.6.1
% see http://writer2latex.sourceforge.net for more info
\documentclass[letterpaper]{article}
\usepackage[latin1]{inputenc}
\usepackage{amsmath}
\usepackage{amssymb,amsfonts,textcomp}
\usepackage[T1]{fontenc}
\usepackage[english]{babel}
\usepackage{color}
\usepackage{array}
\usepackage{supertabular}
\usepackage{hhline}
\usepackage{hyperref}
\usepackage{caption}
\hypersetup{pdftex, colorlinks=true, linkcolor=blue, citecolor=blue, filecolor=blue, urlcolor=blue, pdftitle=, pdfauthor=Barreiro; Andrea, pdfsubject=, pdfkeywords=}
\usepackage[pdftex]{graphicx}

% Text styles
\newcommand\textstyleInternetlink[1]{\textcolor[rgb]{0.019607844,0.3882353,0.75686276}{#1}}
\newcommand\textstylenormaltextrun[1]{#1}
\newcommand\textstyleeop[1]{#1}
% Page layout (geometry)
\setlength\voffset{-1in}
\setlength\hoffset{-1in}
\setlength\topmargin{0.5in}
\setlength\oddsidemargin{1in}
\setlength\textheight{8.079599in}
\setlength\textwidth{6.5in}
\setlength\footskip{0.9602in}
\setlength\headheight{0.5in}
\setlength\headsep{0.4602in}
% Footnote rule
\setlength{\skip\footins}{0.0469in}
\renewcommand\footnoterule{\vspace*{-0.0071in}\setlength\leftskip{0pt}\setlength\rightskip{0pt plus 1fil}\noindent\textcolor{black}{\rule{0.25\columnwidth}{0.0071in}}\vspace*{0.0398in}}
% Pages styles
\makeatletter

\newcommand{\VAR}[1] {$#1$}

\newcommand\ps@Standard{
  \renewcommand\@oddhead{}
  \renewcommand\@evenhead{\@oddhead}
  \renewcommand\@oddfoot{}
  \renewcommand\@evenfoot{\@oddfoot}
  \renewcommand\thepage{\arabic{page}}
}
\makeatother
\pagestyle{Standard}
\setlength\tabcolsep{1mm}
\renewcommand\arraystretch{1.3}
% List styles
\newcommand\liststyleWWNumii{%
\renewcommand\labelitemi{[F0B7?]}
\renewcommand\labelitemii{o}
\renewcommand\labelitemiii{[F0A7?]}
\renewcommand\labelitemiv{[F0B7?]}
}
\newcommand\liststyleWWNumiii{%
\renewcommand\labelitemi{[F0B7?]}
\renewcommand\labelitemii{o}
\renewcommand\labelitemiii{[F0A7?]}
\renewcommand\labelitemiv{[F0B7?]}
}
\newcommand\liststyleWWNumiv{%
\renewcommand\labelitemi{[F0B7?]}
\renewcommand\labelitemii{o}
\renewcommand\labelitemiii{[F0A7?]}
\renewcommand\labelitemiv{[F0B7?]}
}
\newcommand\liststyleWWNumi{%
\renewcommand\labelitemi{{\textperiodcentered}}
\renewcommand\labelitemii{o}
\renewcommand\labelitemiii{[F0A7?]}
\renewcommand\labelitemiv{[F0B7?]}
}

%% This is a template for generating automatic reports for MUMTX.
%% << Say more stuff here >>
%% Here are variables that should be defined (will be referenced as \VAR{variable_name}):
%%   plan_name:       chamber+number short-hand (i.e. S2168 for TX Senate plan with this code on TLV website, 
%%                    or DCN#### for a Dallas City Council plan
%%   number_plans:    number of plans included in ensemble analysis
%%   
%%   plots_directory: directory where plots are to be found
%%   report_directory: directory where low-res plan maps are found
%%   election:        [PRES/SEN] which election to use for first introduction of violin plots
%%   
%%   



\title{}
\author{Barreiro, Andrea}
%\date{2021-10-13}
\begin{document}
\textbf{\textit{MUM\_TX Statement on \VAR{plan_name}}}

\bigskip

%Tuesday, October 12, 2021
\textbf{\textit{Math For Unbiased Maps TX (MUM\_TX) }}is\textbf{\textit{ }}an interdisciplinary, nonpartisan coalition
of Texas mathematicians, political scientists and philosophers working to ensure a fair and transparent redistricting
process. Our research concerns the development and application of ensemble sampling techniques, and in particular their
application to the current TX redistricting cycle. In brief, we use \textit{Markov Chain Monte Carlo }techniques to
generate a large number of random, legally valid maps which can then be used as an unbiased baseline to understand what
a typical map should look like. Conversely, when a proposed map is an outlier from the ensemble, this may be an
indication of gerrymandering.

\bigskip

We have applied our methods to the Congressional maps that have been made available by the Texas Legislative Council. 
%As of 10/12/21, we have seen a large number of maps posted, culminating in \VAR{plan_name} which is now being considered by the House. 
We generated a table of two important statistics that are commonly used by political scientists to assess
partisan gerrymandering: the mean-median score and partisan bias score. \ You can find the table at our webpage:
\href{http://www.smu.edu/Dedman/Research/Institutes-and-Centers/DCII/Scholarship/Research-Cluster-on-Political-Decision-Making/TXGerryWatch}{\textstyleInternetlink{www.smu.edu/Dedman/Research/Institutes-and-Centers/DCII/Scholarship/Research-Cluster-on-Political-Decision-Making/TXGerryWatch}}.

%Unfortunately, our assessment of 10/12/21 is not complementary: \VAR{plan_name} is (1) egregiously gerrymandered to reduce the competitiveness of nearly every congressional district, and (2) manipulated to give the Republican Party, in
%particular, an outsized advantage, completely unlike any plan in our unbiased ensemble. The result is that nearly every
%district (both Republican AND Democratic) is uncompetitive in a general election, and that among these there are far
%too few Democratic districts, given the actual political leanings of Texan voters. Finally (3) minority voters are
%``cracked and packed'' into a few overwhelmingly minority districts, at the cost of decreasing the total number of
%districts which would allow minority communities to elect candidates of their choice.


\bigskip

% We focus on \VAR{plan_name}, which is the map currently under consideration. 
We compared the proposed map to an \textit{ensemble} of \VAR{number_plans} randomly-drawn maps. \ \ \ In Figure~\ref{fig:1}, districts are ordered by the number of votes a Democratic
candidate for US Congress would have received in the 2020 election, had voters used ``straight ticket'' voting. \ On
average, maps within our ensemble (blue dots) exhibit smoothly increasing vote shares as one moves from
Republican-leaning to Democratic-leaning districts. \ This smooth increase is the hallmark of an unbiased map. 
%\ \ But in the proposed map (red dots), the increase is highly disjointed, a clear sign of gerrymandering.

\begin{figure}
\hspace*{-1.5cm}
\includegraphics[width=7.3307in,height=3.9472in]{\VAR{plots_directory}seats-voteshares-ensemble-comps-\VAR{chamber}-\VAR{original_plan_mask}-\VAR{plan}-\VAR{election}20.pdf}
\caption{}
\label{fig:1}
\end{figure}

\bigskip

\begin{figure*}
{\centering \includegraphics[width=6.5in,height=3.5in]{\VAR{plots_directory}seats-voteshares-ensemble-enacted-\VAR{chamber}-\VAR{plan}-\VAR{election}20.pdf} \par}
\caption{}
\label{fig:2}
\end{figure*}

In Figure~\ref{fig:2}, we show the same data, but now identify two signatures of gerrymandering: ``cracking'' (where opposition voters are diluted to create safe districts for the majority) and ``packing'' (where these voters are concentrated to dilute their overall voting power).

\bigskip

% We note several specific features of the proposed plan. First, Democratic voters are disproportionately removed from a
% swath of districts in between 10 and 15 (District numbers are along the x-axis) that would be competitive in an
% unbiased map (a process known as ``cracking''), and placed into uncompetitive districts such as 7, 32, 20, 29, 35, 18,
% and 33 (a process known as ``packing''). \ Second, the list of outcomes between Districts 2 and 3 (a total of 13
% districts) is very nearly flat, which is a hallmark of maps created with the assistance of computer algorithms designed
% to automate the gerrymandering process. \ Finally, the predicted vote share between Districts 15 and 7 changes abruptly
% by about \textit{20 points, }with only 2 districts in between (28 and 34)\textit{ }{}-{}- this represents a ``wall''
% designed to protect legislators from changing voter opinions over time (see actual district numbers on the previous
% figure).

We also compute two common numbers that political scientists use to ``score'' maps. \ The first such number is called
the ``mean-median'' score: the difference in statewide vote percentage between the Republicans and Democrats required for them 
to win the majority of the chamber. %\ For the proposed map, the Republican Party would need to win only \textbf{42.2\%} of the vote to win 19
%seats, while the Democratic Party would need to earn \textbf{57.8\%}; the difference of these numbers gives a
%``mean-median'' score of \textbf{15.6 }(note: to get these numbers from the figure, scale up by a factor of 100). 
The second such score is called the ``partisan bias'' score: \ the difference in the number of seats each party wins if
each were to earn 50\% of the vote. A positive score in either favors Democrats.
%\ For the proposed map, the Republican Party would win \textbf{24 seats} with 50\%
%of the vote, while the Democratic Party would win only \textbf{14 seats}; the difference of these numbers gives a
%``partisan bias'' score of \textbf{{}--10}. \textstylenormaltextrun{In contrast, the median map in our ensemble has a
%mean-median score of \textbf{{}--2.2} and a partisan bias of \textbf{2} (when the statewide vote splits 50-50, the
%Republican Party wins \textbf{18 }seats to the Democratic Party's \textbf{20}).}\textstyleeop{~}

\begin{figure*}
{\centering  \includegraphics[width=5.8154in,height=3.1311in]{\VAR{plots_directory}mean-median-partisan-bias-ensemble-enacted-\VAR{chamber}-\VAR{plan}-\VAR{election}20.pdf}  \\
 \includegraphics[width=5.6909in,height=4.3752in]{\VAR{plots_directory}partisan-metrics-2D-\VAR{chamber}-\VAR{plan}-\VAR{election}20.pdf} \par}
\caption{}
\label{fig:3}
\end{figure*}


\bigskip


Of course, no plan is going to be perfectly aligned with the ensemble, so just how gerrymandered is this plan? \ A
little? \ A lot? \ An extreme amount? \ This question can be answered using statistics, by comparing each score above
to the \textit{distribution }of those scores within the \VAR{number_plans} - map ensemble. \ This is done in Figure~\ref{fig:3}.  The second row shows the joint distribution of mean-median and partisan bias. 

\begin{figure*}
{\centering \includegraphics[width=6.5in,height=3.5in]{\VAR{plots_directory}violin-plot-\VAR{chamber}-\VAR{original_plan_mask}-\VAR{plan}-BH-VAP.pdf} \par}
\caption{}
\label{fig:4}
\end{figure*}

\bigskip

In Figure~\ref{fig:4} we present another ``violin'' plot, but with districts sorted according to the
the fraction of the voting age population that is Black + Hispanic. 

%This is vividly illustrated by the 2D histogram showing the joint distribution of these scores across the ensemble (the
%plot has been smoothed for visualization, as partisan bias is integer-valued). Another evident fact from this histogram
%is that there is no good reason why the legislature must pick such a manipulated map; there are literally hundreds of
%thousands of maps that are less biased.

%, and the results are disappointing. \ As shown in the following figure, both the ``mean-median'' and ``partisan bias''
% scores are very far from their typical values within an unbiased ensemble. \ In fact, both the mean-median and the
% partisan bias scores were more extreme than any value we saw in our ensemble. \textbf{Not a single map} in our ensemble
% had a mean-median score greater than that of the proposed map, and \textbf{not a single map} had a partisan bias score
% as negative. 

%\ \ We clearly observe the same type of trend as in the
%sorting by parties: \ some districts (19, 8, 6) all contain many fewer Black + Hispanic voters than would be expected
%from an unbiased map, whereas others (23, 9, 30) all contain many more Black + Hispanic voters than would be expected.
%\ The story is essentially identical to that observed above -- SB6 removes Black and Hispanic voters from districts
%where they form a near majority, and their voices might influence election outcomes, into districts where they already
%form a large majority.}


\bigskip

We next ask the question ``How many Districts have an BHVAP over 50\%?'' (or 60\%, or 70\%, etc..). The
BHVAP is the combined Black and Hispanic voting age population. These histograms show the values for the ensemble, and
the value for the Proposed map is shown in red. In Figure~\ref{fig:5} we show the number of districts that have BHVAP above 50\% and 70\%; that is the number of districts that are majority-minority, vs. the number of Districts that are
\textbf{\textit{overwhelmingly}} majority-minority. 

\begin{figure*}
\begin{flushleft}
\tablefirsthead{}
\tablehead{}
\tabletail{}
\tablelasttail{}
\begin{supertabular}{|l|l|}
\hline
\selectlanguage{english}  \includegraphics[width=3.2902in,height=2.1945in]{\VAR{plots_directory}hist-small-\VAR{chamber}-\VAR{original_plan}-\VAR{plan}-BH-VAP-0.5.pdf}  &
\selectlanguage{english}  \includegraphics[width=3.2902in,height=2.1945in]{\VAR{plots_directory}hist-small-\VAR{chamber}-\VAR{original_plan}-\VAR{plan}-BH-VAP-0.7.pdf} \\\hline
\end{supertabular}
\end{flushleft}
\caption{}
\label{fig:5}
\end{figure*}
\maxdeadcycles=1000
\extrafloats{100}

%Here is what we observe:

%\liststyleWWNumi
%\begin{itemize}
%\item The number of majority-minority districts in the Proposed plan (\VAR{plan_name}) is much lower than the typical %value in the
%ensemble (14 vs. 16-17).
%\item The number of overwhelmingly majority-minority districts is higher than the typical value in the ensemble (10 vs.
%6).
%\item This suggest that minority voters have been packed into a small number of districts, at the cost of reducing the
%total number of districts in which they may be able to elect representatives of their choice.
%\end{itemize}

%In summary, \VAR{plan_name} fails Texas voters by gerrymandering along both partisan and racial dimensions. First, this map
%artificially reduces the competitiveness of a large number of districts: \textit{at most} 3 out of 38 districts might
%charitably be viewed as competitive in a general election. The map also inflates the advantage to the Republican Party,
%in comparison to a typical unbiased map; an unbiased map would be closely balanced between the parties (and far more
%reflective of the views of actual Texas voters, who voted 54\% to 46\% in the 2020 election). Second, this map packs
%Black and Hispanic voters into a relatively small number of overwhelmingly minority districts, at the cost of reducing
%the total number of districts which effectively perform for minorities. We urge legislators to go back to the drawing
%board and return with a map that is fair to Texas voters.


\bigskip

\section*{Appendix: All Plots}

\begin{figure}
{\centering \includegraphics[width=7.3307in,height=7in]{\VAR{report_directory}report_\VAR{plan_name}_diff_map.png} \par}
\caption{\href{\VAR{url_root}report_\VAR{plan_name}_diff_map.pdf}{Click here} to download a higher resolution version of this image.}
\label{fig:plan_map}
\end{figure}

\begin{figure}
{\centering \includegraphics[width=7.3307in,height=3.5in]{\VAR{plots_directory}seats-voteshares-ensemble-comps-\VAR{chamber}-\VAR{original_plan_mask}-\VAR{plan}-PRES20.pdf} \par}
{\centering \includegraphics[width=7.3307in,height=3.5in]{\VAR{plots_directory}seats-voteshares-ensemble-comps-\VAR{chamber}-\VAR{original_plan_mask}-\VAR{plan}-SEN20.pdf} \par}
\caption{Vote-share vectors for all districts, ordered by increasing Democratic vote share, for the 2020 Presidential (top) and US Senate (bottom) elections. Along with the ensemble (blue violins), vote-share vectors for the proposed (red) and current (green) plans are shown.}
\label{fig:voteshares_ensem_comps}
\end{figure}

\begin{figure}
{\centering \includegraphics[width=6.5in,height=3.5in]{\VAR{plots_directory}seats-voteshares-ensemble-enacted-\VAR{chamber}-\VAR{plan}-PRES20.pdf} \par}
\caption{}
\label{fig:seats-voteshares-ensemble-enacted-PRES20}
\end{figure}

\begin{figure}
{\centering \includegraphics[width=5.8154in,height=3.1311in]{\VAR{plots_directory}mean-median-partisan-bias-ensemble-enacted-\VAR{chamber}-\VAR{plan}-PRES20.pdf} \par}
\caption{}
\label{fig:mean-median-partisan-bias-ensemble-enacted-PRES20}
\end{figure}

\begin{figure}
{\centering \includegraphics[width=4.6909in,height=3.3752in]{\VAR{plots_directory}partisan-metrics-2D-\VAR{chamber}-\VAR{plan}-PRES20.pdf} \par}
\caption{}
\label{fig:partisan-metrics-2D-PRES20}
\end{figure}

\begin{figure}
{\centering \includegraphics[width=6.5in,height=3.5in]{\VAR{plots_directory}seats-voteshares-ensemble-enacted-\VAR{chamber}-\VAR{plan}-SEN20.pdf} \par}
\caption{}
\label{fig:seats-voteshares-ensemble-enacted-SEN20}
\end{figure}

\begin{figure}
{\centering \includegraphics[width=5.8154in,height=3.1311in]{\VAR{plots_directory}mean-median-partisan-bias-ensemble-enacted-\VAR{chamber}-\VAR{plan}-SEN20.pdf} \par}
\caption{}
\label{fig:}mean-median-partisan-bias-ensemble-enacted-SEN20
\end{figure}

\begin{figure}
{\centering \includegraphics[width=4.6909in,height=3.3752in]{\VAR{plots_directory}partisan-metrics-2D-\VAR{chamber}-\VAR{plan}-SEN20.pdf} \par}
\caption{}
\label{fig:partisan-metrics-2D-SEN20}
\end{figure}

\BLOCK{ for population_group in population_groups }

\begin{figure}
{\centering \includegraphics[width=6.5in,height=3.5in]{\VAR{plots_directory}violin-plot-\VAR{chamber}-\VAR{original_plan_mask}-\VAR{plan}-B-\VAR{population_group}.pdf} \par}
\caption{}
\label{fig:}
\end{figure}

\begin{figure}
{\centering \includegraphics[width=6.5in,height=3.5in]{\VAR{plots_directory}violin-plot-\VAR{chamber}-\VAR{original_plan_mask}-\VAR{plan}-H-\VAR{population_group}.pdf} \par}
\caption{}
\label{fig:}
\end{figure}

\begin{figure}
{\centering \includegraphics[width=6.5in,height=3.5in]{\VAR{plots_directory}violin-plot-\VAR{chamber}-\VAR{original_plan_mask}-\VAR{plan}-BH-\VAR{population_group}.pdf} \par}
\caption{}
\label{fig:}
\end{figure}

\begin{figure}
{\centering \includegraphics[width=6.5in,height=3.5in]{\VAR{plots_directory}violin-plot-\VAR{chamber}-\VAR{original_plan_mask}-\VAR{plan}-NW-\VAR{population_group}.pdf} \par}
\caption{}
\label{fig:}
\end{figure}
\BLOCK{ endfor }

\BLOCK{ for population_group in population_groups }
\begin{figure}[h]
\includegraphics[width=3.111in,height=2.0555in]{\VAR{plots_directory}hist-small-\VAR{chamber}-\VAR{original_plan}-\VAR{plan}-B-\VAR{population_group}-0.3.pdf}
\includegraphics[width=3.111in,height=2.0555in]{\VAR{plots_directory}hist-small-\VAR{chamber}-\VAR{original_plan}-\VAR{plan}-B-\VAR{population_group}-0.4.pdf}\\
\includegraphics[width=3.111in,height=2.0555in]{\VAR{plots_directory}hist-small-\VAR{chamber}-\VAR{original_plan}-\VAR{plan}-B-\VAR{population_group}-0.5.pdf}
\caption{}
\label{fig:}
\end{figure}

\begin{figure}[h]
\includegraphics[width=3.111in,height=2.0555in]{\VAR{plots_directory}hist-small-\VAR{chamber}-\VAR{original_plan}-\VAR{plan}-H-\VAR{population_group}-0.5.pdf}
\includegraphics[width=3.111in,height=2.0555in]{\VAR{plots_directory}hist-small-\VAR{chamber}-\VAR{original_plan}-\VAR{plan}-H-\VAR{population_group}-0.55.pdf}\\
\includegraphics[width=3.111in,height=2.0555in]{\VAR{plots_directory}hist-small-\VAR{chamber}-\VAR{original_plan}-\VAR{plan}-H-\VAR{population_group}-0.6.pdf}
\includegraphics[width=3.111in,height=2.0555in]{\VAR{plots_directory}hist-small-\VAR{chamber}-\VAR{original_plan}-\VAR{plan}-H-\VAR{population_group}-0.65.pdf}
\end{figure}

\begin{figure}[h]
\includegraphics[width=3.111in,height=2.0555in]{\VAR{plots_directory}hist-small-\VAR{chamber}-\VAR{original_plan}-\VAR{plan}-BH-\VAR{population_group}-0.5.pdf}
\includegraphics[width=3.111in,height=2.0555in]{\VAR{plots_directory}hist-small-\VAR{chamber}-\VAR{original_plan}-\VAR{plan}-BH-\VAR{population_group}-0.6.pdf}\\
\includegraphics[width=3.111in,height=2.0555in]{\VAR{plots_directory}hist-small-\VAR{chamber}-\VAR{original_plan}-\VAR{plan}-BH-\VAR{population_group}-0.7.pdf}
\caption{}
\label{fig:}
\end{figure}

\begin{figure}[h]
\includegraphics[width=3.111in,height=2.0555in]{\VAR{plots_directory}hist-small-\VAR{chamber}-\VAR{original_plan}-\VAR{plan}-NW-\VAR{population_group}-0.5.pdf}
\includegraphics[width=3.111in,height=2.0555in]{\VAR{plots_directory}hist-small-\VAR{chamber}-\VAR{original_plan}-\VAR{plan}-NW-\VAR{population_group}-0.6.pdf}\\
\includegraphics[width=3.111in,height=2.0555in]{\VAR{plots_directory}hist-small-\VAR{chamber}-\VAR{original_plan}-\VAR{plan}-NW-\VAR{population_group}-0.7.pdf}
\caption{}
\label{fig:}
\end{figure}
\BLOCK{ endfor }

%% VRA Metrics from MGGG
\BLOCK{ if have_vra }
\section*{VRA}

\BLOCK{ for model in ['equal', 'state'] }
\BLOCK{ for racial_group in ['BA', 'B', 'HA', 'H', 'BHO', 'BH'] }
\begin{figure}[h]
\includegraphics[width=6.5in,height=3.5in]{\VAR{plots_directory}violin-plot-\VAR{chamber}-\VAR{original_plan_mask}-\VAR{plan}-\VAR{racial_group}-CVAP-mggg_eff_\VAR{model}.pdf}
\caption{}
\label{fig:}
\end{figure}

\BLOCK{ endfor }
\BLOCK{ endfor }

\BLOCK{ for model in ['equal', 'state'] }
\BLOCK{ for racial_group in ['BA', 'B', 'HA', 'H', 'BHO', 'BH'] }
\begin{figure}[h]
\includegraphics[width=3.111in,height=2.0555in]{\VAR{plots_directory}hist-small-\VAR{chamber}-\VAR{original_plan}-\VAR{plan}-\VAR{racial_group}-CVAP-mggg_eff_\VAR{model}-0.3.pdf}
\includegraphics[width=3.111in,height=2.0555in]{\VAR{plots_directory}hist-small-\VAR{chamber}-\VAR{original_plan}-\VAR{plan}-\VAR{racial_group}-CVAP-mggg_eff_\VAR{model}-0.4.pdf}
\caption{}
\label{fig:}
\end{figure}

\begin{figure}[h]
\includegraphics[width=3.111in,height=2.0555in]{\VAR{plots_directory}hist-small-\VAR{chamber}-\VAR{original_plan}-\VAR{plan}-\VAR{racial_group}-CVAP-mggg_eff_\VAR{model}-0.5.pdf}
\includegraphics[width=3.111in,height=2.0555in]{\VAR{plots_directory}hist-small-\VAR{chamber}-\VAR{original_plan}-\VAR{plan}-\VAR{racial_group}-CVAP-mggg_eff_\VAR{model}-0.6.pdf}
\caption{}
\label{fig:}
\end{figure}

\begin{figure}[h]
\includegraphics[width=3.111in,height=2.0555in]{\VAR{plots_directory}hist-small-\VAR{chamber}-\VAR{original_plan}-\VAR{plan}-\VAR{racial_group}-CVAP-mggg_eff_\VAR{model}-0.7.pdf}
\caption{}
\label{fig:}
\end{figure}
\BLOCK{ endfor }
\BLOCK{ endfor }
\BLOCK{ endif }  %% IF using VRA metrics from MGGG

\BLOCK{ for population_group in population_groups }
\begin{figure}[h]{\centering \includegraphics[width=7.3307in,height=3.9472in]{\VAR{plots_directory}racial-political-deviations-with-ensemble-\VAR{chamber}-\VAR{plan}-PRES20-B-\VAR{population_group}.pdf} \par}
\caption{}
\label{fig:}
\end{figure}
\begin{figure}[h]{\centering \includegraphics[width=7.3307in,height=3.9472in]{\VAR{plots_directory}racial-political-deviations-with-ensemble-\VAR{chamber}-\VAR{plan}-PRES20-H-\VAR{population_group}.pdf} \par}
\caption{}
\label{fig:}
\end{figure}
\begin{figure}[h]{\centering \includegraphics[width=7.3307in,height=3.9472in]{\VAR{plots_directory}racial-political-deviations-with-ensemble-\VAR{chamber}-\VAR{plan}-PRES20-BH-\VAR{population_group}.pdf} \par}
\caption{}
\label{fig:}
\end{figure}
\begin{figure}[h]{\centering \includegraphics[width=7.3307in,height=3.9472in]{\VAR{plots_directory}racial-political-deviations-with-ensemble-\VAR{chamber}-\VAR{plan}-PRES20-NW-\VAR{population_group}.pdf} \par}
\caption{}
\label{fig:}
\end{figure}
\BLOCK{ endfor }

\BLOCK{ for population_group in population_groups }
\begin{figure}[h]{\centering \includegraphics[width=7.3307in,height=3.9472in]{\VAR{plots_directory}racial-political-deviations-with-ensemble-\VAR{chamber}-\VAR{plan}-SEN20-B-\VAR{population_group}.pdf} \par}
\caption{}
\label{fig:}
\end{figure}
\begin{figure}[h]{\centering \includegraphics[width=7.3307in,height=3.9472in]{\VAR{plots_directory}racial-political-deviations-with-ensemble-\VAR{chamber}-\VAR{plan}-SEN20-H-\VAR{population_group}.pdf} \par}
\caption{}
\label{fig:}
\end{figure}
\begin{figure}[h]{\centering \includegraphics[width=7.3307in,height=3.9472in]{\VAR{plots_directory}racial-political-deviations-with-ensemble-\VAR{chamber}-\VAR{plan}-SEN20-BH-\VAR{population_group}.pdf} \par}
\caption{}
\label{fig:}
\end{figure}
\begin{figure}[h]{\centering \includegraphics[width=7.3307in,height=3.9472in]{\VAR{plots_directory}racial-political-deviations-with-ensemble-\VAR{chamber}-\VAR{plan}-SEN20-NW-\VAR{population_group}.pdf} \par}
\caption{}
\label{fig:}
\end{figure}
\BLOCK{ endfor }

\end{document}
